% Options for packages loaded elsewhere
\PassOptionsToPackage{unicode}{hyperref}
\PassOptionsToPackage{hyphens}{url}
%
\documentclass[
]{article}
\usepackage{amsmath,amssymb}
\usepackage{lmodern}
\usepackage{ifxetex,ifluatex}
\ifnum 0\ifxetex 1\fi\ifluatex 1\fi=0 % if pdftex
  \usepackage[T1]{fontenc}
  \usepackage[utf8]{inputenc}
  \usepackage{textcomp} % provide euro and other symbols
\else % if luatex or xetex
  \usepackage{unicode-math}
  \defaultfontfeatures{Scale=MatchLowercase}
  \defaultfontfeatures[\rmfamily]{Ligatures=TeX,Scale=1}
\fi
% Use upquote if available, for straight quotes in verbatim environments
\IfFileExists{upquote.sty}{\usepackage{upquote}}{}
\IfFileExists{microtype.sty}{% use microtype if available
  \usepackage[]{microtype}
  \UseMicrotypeSet[protrusion]{basicmath} % disable protrusion for tt fonts
}{}
\makeatletter
\@ifundefined{KOMAClassName}{% if non-KOMA class
  \IfFileExists{parskip.sty}{%
    \usepackage{parskip}
  }{% else
    \setlength{\parindent}{0pt}
    \setlength{\parskip}{6pt plus 2pt minus 1pt}}
}{% if KOMA class
  \KOMAoptions{parskip=half}}
\makeatother
\usepackage{xcolor}
\IfFileExists{xurl.sty}{\usepackage{xurl}}{} % add URL line breaks if available
\IfFileExists{bookmark.sty}{\usepackage{bookmark}}{\usepackage{hyperref}}
\hypersetup{
  pdftitle={ISSC: Building a personal website workshop series},
  pdfauthor={Liza Bolton},
  hidelinks,
  pdfcreator={LaTeX via pandoc}}
\urlstyle{same} % disable monospaced font for URLs
\usepackage[margin=1in]{geometry}
\usepackage{longtable,booktabs,array}
\usepackage{calc} % for calculating minipage widths
% Correct order of tables after \paragraph or \subparagraph
\usepackage{etoolbox}
\makeatletter
\patchcmd\longtable{\par}{\if@noskipsec\mbox{}\fi\par}{}{}
\makeatother
% Allow footnotes in longtable head/foot
\IfFileExists{footnotehyper.sty}{\usepackage{footnotehyper}}{\usepackage{footnote}}
\makesavenoteenv{longtable}
\usepackage{graphicx}
\makeatletter
\def\maxwidth{\ifdim\Gin@nat@width>\linewidth\linewidth\else\Gin@nat@width\fi}
\def\maxheight{\ifdim\Gin@nat@height>\textheight\textheight\else\Gin@nat@height\fi}
\makeatother
% Scale images if necessary, so that they will not overflow the page
% margins by default, and it is still possible to overwrite the defaults
% using explicit options in \includegraphics[width, height, ...]{}
\setkeys{Gin}{width=\maxwidth,height=\maxheight,keepaspectratio}
% Set default figure placement to htbp
\makeatletter
\def\fps@figure{htbp}
\makeatother
\setlength{\emergencystretch}{3em} % prevent overfull lines
\providecommand{\tightlist}{%
  \setlength{\itemsep}{0pt}\setlength{\parskip}{0pt}}
\setcounter{secnumdepth}{5}
\usepackage{booktabs}
\ifluatex
  \usepackage{selnolig}  % disable illegal ligatures
\fi
\usepackage[]{natbib}
\bibliographystyle{plainnat}

\title{ISSC: Building a personal website workshop series}
\author{Liza Bolton}
\date{Last updated 2021-05-22}

\begin{document}
\maketitle

{
\setcounter{tocdepth}{2}
\tableofcontents
}
\hypertarget{introduction}{%
\section{Introduction}\label{introduction}}

There are many, many ways to make a website these days. This workshop series aims to help you make a personal website using the \href{https://rstudio.github.io/distill/}{\texttt{distill} package}.This is just ONE way that is used by a lots of folks in statistics and that leverages your existing R and R Markdown skills. It is reasonably beginner friendly and creates a professional looking website.

\hypertarget{what-were-going-to-do}{%
\subsection{What we're going to do}\label{what-were-going-to-do}}

\begin{itemize}
\item ~
  \hypertarget{prerequisites}{%
  \subsection{Prerequisites}\label{prerequisites}}
\item
  Build a basic
\end{itemize}

\hypertarget{part-prerequisites}{%
\part{Prerequisites}\label{part-prerequisites}}

\hypertarget{prereqs}{%
\section{What you need before you start}\label{prereqs}}

To be able to engage fully with this series of workshops, you will need to have completed the steps in the Installation part of
\url{https://happygitwithr.com/}. This a great a resource.

\begin{itemize}
\tightlist
\item
  Register a GitHub account
\item
  Get GitHub for education
  \url{https://education.github.com/} so you can have free private
  repositories
\item
  Install or upgrade R and
  RStudio on your local machine
\item
  Install Git
\item
  Introduce yourself to Git
\end{itemize}

\hypertarget{what-is-this-and-why-do-you-want-it}{%
\subsection{What is this and why do you want it?}\label{what-is-this-and-why-do-you-want-it}}

We're suggesting that if you haven't already, you get yourself setup with GitHub as a key component of your portfolio building strategy, regardless of your plans to make a website with this series or otherwise. It will supercharge your version control and your ability to collaborate with others AND provides a FREE way to host a website.

Jenny Bryan has a great introduction in her \href{https://happygitwithr.com}{Happy Git with R}, so I'll let her explain the rest:

Git is a version control system. Its original purpose was to help groups of developers work collaboratively on big software projects. Git manages the evolution of a set of files -- called a repository -- in a sane, highly structured way. If you have no idea what I'm talking about, think of it as the ``Track Changes'' features from Microsoft Word on steroids.
Git has been re-purposed by the data science community. In addition to using it for source code, we use it to manage the motley collection of files that make up typical data analytical projects, which often consist of data, figures, reports, and, yes, source code.
A solo data analyst, working on a single computer, will benefit from adopting version control. But not nearly enough to justify the pain of installation and workflow upheaval. There are much easier ways to get versioned back ups of your files, if that's all you're worried about.
In my opinion, for new users, the pros of Git only outweigh the cons when you factor in the overhead of communicating and collaborating with other people. Who among us does not need to do that? Your life is much easier if this is baked into your workflow, as opposed to being a separate process that you dread or neglect.'' - Jenny Bryan, Happy Git with R, Section 1.1: Why Git? \url{https://happygitwithr.com/big-picture.html}
There is also lots of great practical professional advice in here, too, like ``Pick a username you will be comfortable revealing to your future boss.'' Save gamerangel420 for Reddit. (The first example I thought of I had to change\ldots it actually was someone's Reddit username)
Work through Jenny Bryan's awesome `Happy Git with R'
Check out the DoSS Toolkit lesson ``Git outta here''
Sign up for GitHub Education Student Developer Pack (access to hundreds of dollars worth of tools and training AND a GitHub pro account while you're a student)

\hypertarget{session-1-build-a-simple-website-with-distill}{%
\section{Session 1: Build a simple website with Distill}\label{session-1-build-a-simple-website-with-distill}}

\hypertarget{to-do-before-the-workshop}{%
\subsection{To do before the workshop}\label{to-do-before-the-workshop}}

\hypertarget{workshop-instructions}{%
\subsection{Workshop instructions}\label{workshop-instructions}}

\hypertarget{post-workshop-tasks}{%
\subsection{Post-workshop tasks}\label{post-workshop-tasks}}

\hypertarget{session-2-blogging-with-distill}{%
\section{Session 2: Blogging with Distill}\label{session-2-blogging-with-distill}}

\hypertarget{to-do-before-the-workshop-1}{%
\subsection{To do before the workshop}\label{to-do-before-the-workshop-1}}

\hypertarget{workshop-instructions-1}{%
\subsection{Workshop instructions}\label{workshop-instructions-1}}

\hypertarget{post-workshop-tasks-1}{%
\subsection{Post-workshop tasks}\label{post-workshop-tasks-1}}

\hypertarget{session-3-content-development}{%
\section{Session 3: Content development}\label{session-3-content-development}}

\hypertarget{to-do-before-the-workshop-2}{%
\subsection{To do before the workshop}\label{to-do-before-the-workshop-2}}

\hypertarget{workshop-instructions-2}{%
\subsection{Workshop instructions}\label{workshop-instructions-2}}

\hypertarget{post-workshop-tasks-2}{%
\subsection{Post-workshop tasks}\label{post-workshop-tasks-2}}

\hypertarget{appendix-appendix}{%
\appendix}


\hypertarget{domain}{%
\section{Buy a domain name (optional)}\label{domain}}

You \textbf{do not have to buy anything} to build and host a professional personal website as a student. BUT, if you've always wanted to own www.firstnamelastname.com, here are some instructions. Note, if you're not sure what a domain name is there are heaps of explainer articles online, just search ``what is a domain name''. You don't need to know too much.

\hypertarget{what-if-you-dont-buy-a-domain-name}{%
\subsubsection{What if you don't buy a domain name?}\label{what-if-you-dont-buy-a-domain-name}}

If you don't want to buy a domain name, your personal website's address will appear as: \texttt{your-github-username.github.io}. Even more reason to choose a good username when you sign up in the \protect\hyperlink{prereqs}{prerequisite secion}.

For example, the slides from the colour palette talk I did live on there own little website: \url{https://uoft-doss-issc.github.io/ggplot-colour-palettes/}

\hypertarget{caveats}{%
\subsection{Caveats}\label{caveats}}

\begin{itemize}
\tightlist
\item
  I'd recommend doing some of your own further research to make sure this is actually the best approach for you and what you actually want or need.
\item
  This isn't legal or financial advice. Duh.
\end{itemize}

\end{document}
