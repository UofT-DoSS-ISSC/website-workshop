% Options for packages loaded elsewhere
\PassOptionsToPackage{unicode}{hyperref}
\PassOptionsToPackage{hyphens}{url}
%
\documentclass[
]{article}
\usepackage{amsmath,amssymb}
\usepackage{lmodern}
\usepackage{ifxetex,ifluatex}
\ifnum 0\ifxetex 1\fi\ifluatex 1\fi=0 % if pdftex
  \usepackage[T1]{fontenc}
  \usepackage[utf8]{inputenc}
  \usepackage{textcomp} % provide euro and other symbols
\else % if luatex or xetex
  \usepackage{unicode-math}
  \defaultfontfeatures{Scale=MatchLowercase}
  \defaultfontfeatures[\rmfamily]{Ligatures=TeX,Scale=1}
\fi
% Use upquote if available, for straight quotes in verbatim environments
\IfFileExists{upquote.sty}{\usepackage{upquote}}{}
\IfFileExists{microtype.sty}{% use microtype if available
  \usepackage[]{microtype}
  \UseMicrotypeSet[protrusion]{basicmath} % disable protrusion for tt fonts
}{}
\makeatletter
\@ifundefined{KOMAClassName}{% if non-KOMA class
  \IfFileExists{parskip.sty}{%
    \usepackage{parskip}
  }{% else
    \setlength{\parindent}{0pt}
    \setlength{\parskip}{6pt plus 2pt minus 1pt}}
}{% if KOMA class
  \KOMAoptions{parskip=half}}
\makeatother
\usepackage{xcolor}
\IfFileExists{xurl.sty}{\usepackage{xurl}}{} % add URL line breaks if available
\IfFileExists{bookmark.sty}{\usepackage{bookmark}}{\usepackage{hyperref}}
\hypersetup{
  pdftitle={ISSC: Building a personal website workshop series},
  pdfauthor={Liza Bolton},
  hidelinks,
  pdfcreator={LaTeX via pandoc}}
\urlstyle{same} % disable monospaced font for URLs
\usepackage[margin=1in]{geometry}
\usepackage{color}
\usepackage{fancyvrb}
\newcommand{\VerbBar}{|}
\newcommand{\VERB}{\Verb[commandchars=\\\{\}]}
\DefineVerbatimEnvironment{Highlighting}{Verbatim}{commandchars=\\\{\}}
% Add ',fontsize=\small' for more characters per line
\usepackage{framed}
\definecolor{shadecolor}{RGB}{248,248,248}
\newenvironment{Shaded}{\begin{snugshade}}{\end{snugshade}}
\newcommand{\AlertTok}[1]{\textcolor[rgb]{0.94,0.16,0.16}{#1}}
\newcommand{\AnnotationTok}[1]{\textcolor[rgb]{0.56,0.35,0.01}{\textbf{\textit{#1}}}}
\newcommand{\AttributeTok}[1]{\textcolor[rgb]{0.77,0.63,0.00}{#1}}
\newcommand{\BaseNTok}[1]{\textcolor[rgb]{0.00,0.00,0.81}{#1}}
\newcommand{\BuiltInTok}[1]{#1}
\newcommand{\CharTok}[1]{\textcolor[rgb]{0.31,0.60,0.02}{#1}}
\newcommand{\CommentTok}[1]{\textcolor[rgb]{0.56,0.35,0.01}{\textit{#1}}}
\newcommand{\CommentVarTok}[1]{\textcolor[rgb]{0.56,0.35,0.01}{\textbf{\textit{#1}}}}
\newcommand{\ConstantTok}[1]{\textcolor[rgb]{0.00,0.00,0.00}{#1}}
\newcommand{\ControlFlowTok}[1]{\textcolor[rgb]{0.13,0.29,0.53}{\textbf{#1}}}
\newcommand{\DataTypeTok}[1]{\textcolor[rgb]{0.13,0.29,0.53}{#1}}
\newcommand{\DecValTok}[1]{\textcolor[rgb]{0.00,0.00,0.81}{#1}}
\newcommand{\DocumentationTok}[1]{\textcolor[rgb]{0.56,0.35,0.01}{\textbf{\textit{#1}}}}
\newcommand{\ErrorTok}[1]{\textcolor[rgb]{0.64,0.00,0.00}{\textbf{#1}}}
\newcommand{\ExtensionTok}[1]{#1}
\newcommand{\FloatTok}[1]{\textcolor[rgb]{0.00,0.00,0.81}{#1}}
\newcommand{\FunctionTok}[1]{\textcolor[rgb]{0.00,0.00,0.00}{#1}}
\newcommand{\ImportTok}[1]{#1}
\newcommand{\InformationTok}[1]{\textcolor[rgb]{0.56,0.35,0.01}{\textbf{\textit{#1}}}}
\newcommand{\KeywordTok}[1]{\textcolor[rgb]{0.13,0.29,0.53}{\textbf{#1}}}
\newcommand{\NormalTok}[1]{#1}
\newcommand{\OperatorTok}[1]{\textcolor[rgb]{0.81,0.36,0.00}{\textbf{#1}}}
\newcommand{\OtherTok}[1]{\textcolor[rgb]{0.56,0.35,0.01}{#1}}
\newcommand{\PreprocessorTok}[1]{\textcolor[rgb]{0.56,0.35,0.01}{\textit{#1}}}
\newcommand{\RegionMarkerTok}[1]{#1}
\newcommand{\SpecialCharTok}[1]{\textcolor[rgb]{0.00,0.00,0.00}{#1}}
\newcommand{\SpecialStringTok}[1]{\textcolor[rgb]{0.31,0.60,0.02}{#1}}
\newcommand{\StringTok}[1]{\textcolor[rgb]{0.31,0.60,0.02}{#1}}
\newcommand{\VariableTok}[1]{\textcolor[rgb]{0.00,0.00,0.00}{#1}}
\newcommand{\VerbatimStringTok}[1]{\textcolor[rgb]{0.31,0.60,0.02}{#1}}
\newcommand{\WarningTok}[1]{\textcolor[rgb]{0.56,0.35,0.01}{\textbf{\textit{#1}}}}
\usepackage{longtable,booktabs,array}
\usepackage{calc} % for calculating minipage widths
% Correct order of tables after \paragraph or \subparagraph
\usepackage{etoolbox}
\makeatletter
\patchcmd\longtable{\par}{\if@noskipsec\mbox{}\fi\par}{}{}
\makeatother
% Allow footnotes in longtable head/foot
\IfFileExists{footnotehyper.sty}{\usepackage{footnotehyper}}{\usepackage{footnote}}
\makesavenoteenv{longtable}
\usepackage{graphicx}
\makeatletter
\def\maxwidth{\ifdim\Gin@nat@width>\linewidth\linewidth\else\Gin@nat@width\fi}
\def\maxheight{\ifdim\Gin@nat@height>\textheight\textheight\else\Gin@nat@height\fi}
\makeatother
% Scale images if necessary, so that they will not overflow the page
% margins by default, and it is still possible to overwrite the defaults
% using explicit options in \includegraphics[width, height, ...]{}
\setkeys{Gin}{width=\maxwidth,height=\maxheight,keepaspectratio}
% Set default figure placement to htbp
\makeatletter
\def\fps@figure{htbp}
\makeatother
\setlength{\emergencystretch}{3em} % prevent overfull lines
\providecommand{\tightlist}{%
  \setlength{\itemsep}{0pt}\setlength{\parskip}{0pt}}
\setcounter{secnumdepth}{5}
\usepackage{booktabs}
\ifluatex
  \usepackage{selnolig}  % disable illegal ligatures
\fi
\usepackage[]{natbib}
\bibliographystyle{plainnat}

\title{ISSC: Building a personal website workshop series}
\author{Liza Bolton}
\date{Last updated 2021-06-04}

\begin{document}
\maketitle

{
\setcounter{tocdepth}{2}
\tableofcontents
}
\textbf{Unless told otherwise, assume everything in this guide is as yet incomplete!}

\hypertarget{introduction}{%
\section{Introduction}\label{introduction}}

There are many, many ways to make a website these days. This workshop series aims to help you make a personal website using the \href{https://rstudio.github.io/distill/}{\texttt{distill} package}. This is just ONE way that is used by a lots of folks in statistics and that leverages your existing R and R Markdown skills. It is reasonably beginner friendly and creates a professional looking website.

Note: If you already have a personal website, sessions 2 and/or 3 may still be of interest to you. You can attend as many or as few of these sessions as you would like.

\hypertarget{why-make-a-website}{%
\subsection{Why make a website?}\label{why-make-a-website}}

There are many reasons why you might want or need a website (which might also be why there are so many ways to make one!). I'm going to assume your main motivations might be some subset of the below:

\begin{itemize}
\tightlist
\item
  You want to make it easier for employers/prospective supervisors to find examples of your analysis and communication work, your projects (i.e., your portfolio).
\item
  You want to have a home for sharing your side projects with folks with similar interests.
\item
  You want to have a hub from which to share your bio, CV other profiles/links etc., a detailed digital business card of sorts.
\item
  It sounds fun!
\end{itemize}

At the core of why I'm imaging you're interested is because on some level you know that \textbf{communication} is important. You want to communicate to the world about who you are and what you can do. Building a website can be useful both as the \emph{medium} for sharing, but also as a \emph{method} of exploring and reflecting on how you want to communicate about yourself to the world. For this reason, this series aims to combine the coding and tools aspect, with other tasks around writing and reflection, so that once you have a website, you'll also be clear on your purpose for it, how it should look and feel, and have some things to put on it.

\hypertarget{what-were-going-to-do-during-the-workshop-series}{%
\subsection{What we're going to do during the workshop series}\label{what-were-going-to-do-during-the-workshop-series}}

\begin{itemize}
\tightlist
\item
  Get set up with R/RStudio/Git and GitHub (if not already)
\item
  Build a basic landing page website with \texttt{postcards}
\item
  Create a multi-paged paged website with \texttt{distill}
\item
  Begin to develop an aesthetic for your "`personal brand' \footnote{Do we hate the concept of `personal brand'? I think there is some value in thinking about how we make it easier for the world to understand us\ldots but there is so much grossness with it too\ldots{}}
\item
  Learn some basic ways to edit the look and feel of a site with CSS
\item
  Set up and post a first blog post
\item
  Develop and polish professional content for your website and blog
\item
  Connect to additional U of T resources that can help you develop your professional digital presence (TBC)
\end{itemize}

In the HTML version of these notes there is a GIF here.

\emph{GIF description: Boy sitting at computer turns to camera and gives a thumbs up and nods his head repeatedly. The name Brent Rambo appears on screen.}

\hypertarget{details}{%
\subsection{Details}\label{details}}

\begin{itemize}
\tightlist
\item
  \protect\hyperlink{sesh1}{Session 1} was held on 2021-06-03 and a recording is available to members of the Independent Summer Statistics Community and Department of Statistical Sciences (University of Toronto) \href{https://utoronto.sharepoint.com/sites/ArtSci-STA/ISSC/SitePages/Past-events.aspx\#website-building-workshop-series-session-1}{on SharePoint}.
\end{itemize}

\hypertarget{part-prerequisites}{%
\part{Prerequisites}\label{part-prerequisites}}

\hypertarget{prereqs}{%
\section{What you need to do before the first session}\label{prereqs}}

\hypertarget{decide-if-this-is-the-right-way-of-making-a-website-for-you}{%
\subsection{Decide if this is the right way of making a website for you}\label{decide-if-this-is-the-right-way-of-making-a-website-for-you}}

In order to make the most of this series, you should think about what the purpose for your personal website is going to be.

\begin{itemize}
\tightlist
\item
  Is it to have a basic hub for your professional links? (A bit like the `linktree' links you might see on Instagram.)
\item
  Is it to share your technical portfolio and/or demonstrate your communication skills with simple blogs? Especially good if R is one of the main languages you'll be using. Also works with Python with the \href{https://rstudio.github.io/reticulate/}{\texttt{reticulate} package}.
\end{itemize}

If yes, to the above, this workshop series using \texttt{distill} is probably perfect for you. If you want a more complicated and customizable site checkout \href{https://bookdown.org/yihui/blogdown/}{\texttt{blogdown}} (still using an R package) or \href{https://www.w3schools.com/howto/howto_website_bootstrap.asp}{Bootstrap} (HTML/CSS) or if you need to support e-commerce, you might just want to go find which YouTuber you like has a Squarespace sponsorship this week\ldots{}

\hypertarget{technical-tasks}{%
\subsection{Technical tasks}\label{technical-tasks}}

To be able to engage fully with this series of workshops, you will need to have completed the steps in the Installation part of
\url{https://happygitwithr.com/}. This a great a resource. I also STRONGLY recommend that you try to follow the instructions in Chapter 10 so you can easily set up your new project and connect it to GitHub.

\begin{itemize}
\tightlist
\item
  Register a GitHub account
\item
  Get GitHub for education
  \url{https://education.github.com/} so you can have free private
  repositories
\item
  \href{https://happygitwithr.com/install-r-rstudio.html}{Install or upgrade R and
  RStudio on your local machine}
\item
  \href{https://happygitwithr.com/install-git.html}{Install Git}
\item
  \href{https://happygitwithr.com/hello-git.html}{Introduce yourself to Git}
\item
  \href{https://happygitwithr.com/credential-caching.html}{Cache your credentials}
\end{itemize}

For the easiest experience with website building and updating, you want to be able to push to a GitHub repo from either the Git pane in RStudio or using the terminal (which you can also access from RStudion, tab next to the console pane).

\hypertarget{what-is-github-and-why-do-you-want-it}{%
\subsubsection{What is GitHub and why do you want it?}\label{what-is-github-and-why-do-you-want-it}}

We're suggesting that if you haven't already, you get yourself setup with GitHub as a key component of your portfolio building strategy, regardless of your plans to make a website with this series or otherwise. It will supercharge your version control and your ability to collaborate with others AND provides a FREE way to host a website.

Jenny Bryan has a great introduction in her \href{https://happygitwithr.com}{Happy Git with R}, so I'll let her explain the rest:

Git is a version control system. Its original purpose was to help groups of developers work collaboratively on big software projects. Git manages the evolution of a set of files -- called a repository -- in a sane, highly structured way. If you have no idea what I'm talking about, think of it as the ``Track Changes'' features from Microsoft Word on steroids.
Git has been re-purposed by the data science community. In addition to using it for source code, we use it to manage the motley collection of files that make up typical data analytical projects, which often consist of data, figures, reports, and, yes, source code.
A solo data analyst, working on a single computer, will benefit from adopting version control. But not nearly enough to justify the pain of installation and workflow upheaval. There are much easier ways to get versioned back ups of your files, if that's all you're worried about.
In my opinion, for new users, the pros of Git only outweigh the cons when you factor in the overhead of communicating and collaborating with other people. Who among us does not need to do that? Your life is much easier if this is baked into your workflow, as opposed to being a separate process that you dread or neglect.'' - Jenny Bryan, Happy Git with R, Section 1.1: Why Git? \url{https://happygitwithr.com/big-picture.html}
There is also lots of great practical professional advice in here, too, like ``Pick a username you will be comfortable revealing to your future boss.'' Save gamerangel420 for Reddit. (The first example I thought of I had to change\ldots it actually was someone's Reddit username)
Work through Jenny Bryan's awesome `Happy Git with R'
Check out the DoSS Toolkit lesson ``Git outta here''
Sign up for GitHub Education Student Developer Pack (access to hundreds of dollars worth of tools and training AND a GitHub pro account while you're a student)

\hypertarget{part-workshop-sessions}{%
\part{Workshop sessions}\label{part-workshop-sessions}}

\hypertarget{sesh1}{%
\section{Session 1: Build a simple website with Distill}\label{sesh1}}

\hypertarget{s1pre}{%
\subsection{To do before the workshop}\label{s1pre}}

\begin{itemize}
\item
  Complete all the steps in the \protect\hyperlink{prereqs}{prerequisites} section.
\item
  Decide if you want to buy a \protect\hyperlink{domain}{domain name (see appendix)}. Buying a domain name is \textbf{totally optional}, and you don't have to decide now. But, if you know you want to, you might as well get set up now.
\item
  Decide what links to external things you want to share on your website. Have them easily accessible to copy and paste. \emph{GitHub? LinkedIn? Twitter?}
\item
  Choose a profile picture for your website. A headshot with a simple background is best. If you don't have one/don't want to show a photo, you may want to find some other form of an avatar or placeholder picture. Have the PNG or JPEG easily available to copy to the file you'll need.
\item
  \href{https://github.com/rstudio/cheatsheets/raw/master/rmarkdown-2.0.pdf}{Download the R Markdown cheat sheet} and pay special attention to the `\textbf{Pandoc's Markdown}' section on the lefthand side of the second page.
\item
  Install the below R packages. I.e., copy and paste the below code into your R console and run it. If you're not sure what this means or how to do it, come talk to use during a TidyTuesday and Talk session (Tuesdays, from 2:00 to 3:00 p.m. ET, on Zoom, see the \href{https://utoronto.sharepoint.com/sites/ArtSci-STA/ISSC/_layouts/15/Events.aspx?ListGuid=0679786c-8a7e-483c-9ec5-3845602a70e5}{Events list})
\end{itemize}

\begin{Shaded}
\begin{Highlighting}[]
\FunctionTok{install.packages}\NormalTok{(}\StringTok{\textquotesingle{}postcards\textquotesingle{}}\NormalTok{) }\CommentTok{\# for landing page}
\FunctionTok{install.packages}\NormalTok{(}\StringTok{\textquotesingle{}distill\textquotesingle{}}\NormalTok{) }\CommentTok{\# for multipage site}
\FunctionTok{install.packages}\NormalTok{(}\StringTok{\textquotesingle{}usethis\textquotesingle{}}\NormalTok{) }\CommentTok{\# for easy connection to Git and GitHub}
\end{Highlighting}
\end{Shaded}

\hypertarget{thank-you-rohan}{%
\subsubsection{Thank you, Rohan!}\label{thank-you-rohan}}

Major credit to Prof.~Rohan Alexander for his teaching notes about making a website. You can find them at: \url{https://www.tellingstorieswithdata.com/interactive-communication.html\#making-a-website}

His personal website is also made with distill, \url{https://rohanalexander.com/}.

\hypertarget{workshop-instructions}{%
\subsection{Workshop instructions}\label{workshop-instructions}}

\hypertarget{landing-page-with-postcards}{%
\subsubsection{\texorpdfstring{Landing page with \texttt{postcards}}{Landing page with postcards}}\label{landing-page-with-postcards}}

Let's start with the most basic thing we can do, create a landing page with a brief bio and some key links. It is a bit like a business card for the digital age.

This page can later become the home page of your website, or you might find you want to do something more customized.

\begin{enumerate}
\def\labelenumi{\arabic{enumi}.}
\item
  Make sure you have \protect\hyperlink{s1pre}{installed packages you needed to.}
\item
  Create a new R project using the postcard template:

  \begin{enumerate}
  \def\labelenumii{\alph{enumii}.}
  \tightlist
  \item
    `File -\textgreater{} New Project -\textgreater{} New Directory -\textgreater{} Postcards Website.'.
  \item
    Choose a sensible location for this project to live on your computer. I'd recommend just calling this something like `task1'.
  \item
    Pick a theme, I'd recommend trying `trestles' for now, but you can see all the different `looks' \href{https://github.com/seankross/postcards}{here (scroll down)}.
  \item
    Choose `Open in new session', just to be safe.
  \end{enumerate}
\item
  Run \texttt{file.create(\textquotesingle{}.nojekyll\textquotesingle{})} in your console. It will later tell GitHub something about how \emph{not} to build your site.
\item
  Open the `index.Rmd' file and Knit it to View what it looks like.
\item
  Change the title to your name, add your photo to the folder you're working in and change the image to that. Update the links with whatever you want to include. Update the text below the YAML. You might like to use the headings ``Education'' and ``Projects'' if you're likely to stick with a one-page site.
\item
  Set up Git and a GitHub repo. Commit and push. There are quite a few ways to do this (see \href{https://happygitwithr.com/existing-github-last.html}{Happy Git with R} to GitHub with the help of the \texttt{usethis} package.

  \begin{enumerate}
  \def\labelenumii{\alph{enumii}.}
  \item
    In your console, run \texttt{usethis::use\_git()} .

    \begin{enumerate}
    \def\labelenumiii{\arabic{enumiii}.}
    \item
      You'll be asked if it is okay to commit your uncommitted files. Read the options and enter the number for `Yes'.
    \item
      You'll then be asked about restarting so the Git panel can be available. Choose to restart.
    \end{enumerate}
  \item
    If you configured your GitHub Personal Access Token (PAT) (see Chapter 10 of Happy Git with R: Cache credentials for HTTPS) you can run \texttt{usethis::use\_github()} in your console to set up and push to a repo of the same name as the folder you set up (e.g.~`task1'). You'll be able to delete this later once your website has taken it's final form.
  \item
    GitHub should open in your browser. If not, navigate to your GitHub repo for the site and go Settings (top horizontal menu) and then Pages (vertical menus on left)
  \item
    Under source, change from None to master (or main if that is what your branch is called) and change /(root) to /docs and click save.
  \end{enumerate}
\item
  Make a change and repush to GitHub using the terminal tab! (Update the commit message.) Or use the Git pane, or however else you've learned to set yourself up.
\end{enumerate}

\begin{verbatim}
git status
git add -A
git commit -m "look at this hopefully meaningful message"
git push
\end{verbatim}

\hypertarget{example}{%
\paragraph{Example}\label{example}}

I made the ISSC a landing page. You can view it at \url{https://uoft-doss-issc.github.io/} and the code is available on \href{https://github.com/UofT-DoSS-ISSC/UofT-DoSS-ISSC.github.io}{GitHub}.

\hypertarget{multipage-site}{%
\subsubsection{Multipage site}\label{multipage-site}}

\begin{enumerate}
\def\labelenumi{\arabic{enumi}.}
\item
  Make sure you have \protect\hyperlink{s1pre}{installed packages you needed to.}
\item
  Create a new R project using the distill website template:

  \begin{enumerate}
  \def\labelenumii{\alph{enumii}.}
  \tightlist
  \item
    `File -\textgreater{} New Project -\textgreater{} New Directory -\textgreater{} Distill Website.'.
  \item
    Choose a sensible location for this project to live on your computer. I'd recommend just calling this something like `task2' for now.
  \item
    Tick `Configure for GitHub pages'.
  \item
    Choose `Open in new session', just to be safe.
  \end{enumerate}
\item
  Navigate to the \textbf{Build} tab in the environment pane (usually the top right pane in your R Studio if you haven't customized the layout) and click \textbf{Build Website}. This should create a pop-up that allows you to preview what the basic template builds.
\item
  Open the \textbf{index.Rmd} file you made in task 1 and copy the entire thing into the \textbf{index.Rmd} file in your current project.
\item
  Add this line to your YAML \texttt{site:\ distill::distill\_website} for this newly transplanted \textbf{index.Rmd.} Doesn't matter which line you put it on.
\item
  Now we're going to start by exploring the \_\textbf{site.yml} file. This is how you'll set up the navigation bar for your website.

  Start by updating the name, title and description. Only change the description on line 4 and keep the tab in front of the text. Wow, look at this creative example!
\end{enumerate}

\begin{verbatim}
    name: "lizawebsite"
    title: "Liza Bolton"
    description: |
    Liza Bolton
\end{verbatim}

Next, let's add links to any social media or other platforms you want to promote. Don't link social media or platforms you wouldn't be comfortable with an employer seeing (and probably make those accounts private). This example is for GitHub. You can also do Twitter (`fa-twitter`) and LinkedIn (`fa-linkedin`) and probably some others.

\begin{verbatim}
    - icon: fa fa-github
      href: https://github.com/elb0
\end{verbatim}

Want to make it easy to email you? Add your email address and a email icon or cute paper plane icon (`fa-paper-plane`) to the navigation bar. Or you could add these as buttons under your picture if you'd prefer.

\begin{verbatim}
   - icon: fa fa-envelope
     href: mailto:liza.bolton@utoronto.ca
\end{verbatim}

\begin{enumerate}
\def\labelenumi{\arabic{enumi}.}
\setcounter{enumi}{6}
\item
  Let's create a \textbf{Projects} page. Run this code in your console: \texttt{distill::create\_article("projects").} This will create a new Rmd call \textbf{projects.Rmd}. Update it, Knit it and then we'll add it to the \textbf{\_site.yml} and build the site again.

  \begin{enumerate}
  \def\labelenumii{\arabic{enumii}.}
  \tightlist
  \item
    Want a table of contents? Update your YAML.
  \end{enumerate}

\begin{verbatim}
output:
    distill::distill_article:
        toc: true
        toc_depth: 3
\end{verbatim}
\item
  Next, we're going to create a CV page. You can either have your CV as a PDF and set it up to open directly or create a new page with your CV information.

  \begin{enumerate}
  \def\labelenumii{\arabic{enumii}.}
  \item
    If you're creating a page, you'll do basically the same things as the step above.
  \item
    If you have a PDF of your CV, create a new folder called \textbf{pdfs} in your main file (you could call it something else too\ldots) and put the PDF you want in there. Then we'll add it to the navigation bar with:
  \end{enumerate}
\end{enumerate}

\begin{verbatim}
    - text: "CV"  
      href: pdfs/mycv.pdf
\end{verbatim}

(But update the file name.)

\hypertarget{other-tipsresources}{%
\subsection{Other tips/resources}\label{other-tipsresources}}

\begin{itemize}
\tightlist
\item
  Need more help with Git/GitHub? Check out the \href{https://dosstoolkit.com/}{DoSS Toolkit tutorial} on it.
\item
  Add \texttt{.DS\_Store} to your \texttt{.gitignore} file (if you're on a Mac).
\item
  Here are the \href{https://rstudio.github.io/distill/website.html}{getting started notes from the distill team}!
\item
  \href{https://blog.rstudio.com/2020/12/07/distill/}{This blog post from the RStudio team} is a great feature summary for the \texttt{distill} package.
\item
  Explore the \href{https://distillery.rbind.io/showcase.html}{Distillery Showcase} of websites made with Distill! Some great inspiration here.
\end{itemize}

\hypertarget{setting-up-a-custom-domain}{%
\subsection{Setting up a custom domain}\label{setting-up-a-custom-domain}}

It can take a while (like, a day or so) to update the redirection between your GitHub page and your domain, so be prepared to be patient!

These instructions are quite good: \url{https://richpauloo.github.io/2019-11-17-Linking-a-Custom-Domain-to-Github-Pages/}

\hypertarget{post-workshop-tasks}{%
\subsection{Post-workshop tasks}\label{post-workshop-tasks}}

\begin{itemize}
\tightlist
\item
  Check-in with yourself: What was fun about today? What did you learn? What did you find challenging? Did anything make you want to give up? If so, was why should you keep going with your website?
\item
  Fill out this \href{https://forms.office.com/Pages/ResponsePage.aspx?id=JsKqeAMvTUuQN7RtVsVSEOKHUU3SzAJJhmOKjJhDWEpUQUJETTYxTkhUVUlQU0VUOEFFMlBWTjA2Vy4u}{short `ticket out the door' survey.}
\item
  Do a social media audit. What are you happy with being public and what is private to close friends/family, etc.
\item
  Continue to update your bio, links.
\item
  Make a site plan. What other pages would you like to add? Maybe a testimonials page?
\item
  Start thinking about what colours you like and/or what colours would help communicate your personality. Currently your site will have a blue navigation bar, but we will learn some basic ways to change the look and feel in our next session.
\item
  Add your website URL to your email signature, once you feel it is in an okay state of readiness. We're also going to add more to it over the next few sessions so feel free to wait, too.
\end{itemize}

\hypertarget{session-2-blogging-with-distill}{%
\section{Session 2: Blogging with Distill}\label{session-2-blogging-with-distill}}

\textbf{More information coming soon.}

\hypertarget{session-3-content-development}{%
\section{Session 3: Content development}\label{session-3-content-development}}

\textbf{More information coming soon.}

\hypertarget{appendix-appendix}{%
\appendix}


\hypertarget{domain}{%
\section{Buy a domain name (optional)}\label{domain}}

You \textbf{do not have to buy anything} to build and host a professional personal website as a student. BUT, if you've always wanted to own www.firstnamelastname.com, here are some instructions. Note, if you're not sure what a domain name is there are heaps of explainer articles online, just search ``what is a domain name''. You don't really need to know too much.

\hypertarget{some-caveats}{%
\subsubsection{Some caveats}\label{some-caveats}}

\begin{itemize}
\tightlist
\item
  I'd recommend doing some of your own further research to make sure this is actually the best approach for you and what you actually want or need.
\item
  This isn't legal or financial advice. Duh.
\end{itemize}

\hypertarget{dont-pay-full-price}{%
\subsubsection{Don't pay full price!}\label{dont-pay-full-price}}

There is almost always a deal going on. May not still be live when you're looking, but `nameboy' was a code for 25\% of domains.com when I was setting up mine.

\hypertarget{but-what-if-you-dont-buy-a-domain-name}{%
\subsection{But what if you don't buy a domain name?}\label{but-what-if-you-dont-buy-a-domain-name}}

If you don't want to buy a domain name, your personal website's address will appear as: \texttt{your-github-username.github.io}. Even more reason to choose a good username! See more in the \protect\hyperlink{prereqs}{prerequisite secion}.

For example, I made a simple landing page for the ISSC at \url{https://uoft-doss-issc.github.io/}.

\end{document}
